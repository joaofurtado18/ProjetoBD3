% #############################################################################
% Preamble for IST-UL-Project-Report in English or Portuguese
% Required Packages and commands
% --> Please Choose the MAIN LANGUAGE for the Thesis in package BABEL (below)
% !TEX root = ./main.tex
% #############################################################################
% Version 1.0, October 2018
% BY: Prof. Rui Santos Cruz, rui.s.cruz@tecnico.ulisboa.pt
% ######################+OPTIONS: tex:t########################################################
% -----------------------------------------------------------------------------
% PACKAGE ucs:
% -----------------------------------------------------------------------------
% The 'ucs' package provides support for using UTF-8 in LaTeX documents. 
% However in most situations it is not required.
\usepackage{ucs}
% -----------------------------------------------------------------------------
% PACKAGE utf8x:
% -----------------------------------------------------------------------------
% The 'utf8x' package contains support for using UTF-8 as input encoding. 
\usepackage[utf8x]{inputenc}
% -----------------------------------------------------------------------------
% PACKAGE babel:
% -----------------------------------------------------------------------------
% The 'babel' package may correct some hyphenation issues of LaTeX. 
% Select your MAIN LANGUAGE for the Thesis with the 'main=' option.
% 
\usepackage[main=english]{babel}
% -----------------------------------------------------------------------------
% PACKAGE iflang:
% -----------------------------------------------------------------------------
% The 'iflang' package is used to help determine the language being used. 
\usepackage{iflang}
% -----------------------------------------------------------------------------
% PACKAGE graphicx:
% -----------------------------------------------------------------------------
% The 'graphicx' package allows that pdf files can be inserted. 
\usepackage{graphicx}

\usepackage{glossaries}
% ----------------------#+OPTIONS: tex:t------------------------------------------------
% Define default and cover page fonts.
% ----------------------------------------------------------------------
% Use Arial font as default
%
\usepackage{mathptmx}

\renewcommand{\rmdefault}{ptm}
\renewcommand{\sfdefault}{ptm}
\def\FontLn{% 16 pt normal
  \usefont{T1}{ptm}{m}{n}\fontsize{16pt}{16pt}\selectfont}
\def\FontLb{% 16 pt bold
  \usefont{T1}{ptm}{b}{n}\fontsize{16pt}{16pt}\selectfont}
\def\FontMn{% 14 pt normal
  \usefont{T1}{ptm}{m}{n}\fontsize{14pt}{14pt}\selectfont}
\def\FontMb{% 14 pt bold
  \usefont{T1}{ptm}{b}{n}\fontsize{14pt}{14pt}\selectfont}
\def\FontSn{% 12 pt normal
  \usefont{T1}{ptm}{m}{n}\fontsize{12pt}{12pt}\selectfont}
% ----------------------------------------------------------------------
% Define page margins and line spacing.
% ----------------------------------------------------------------------
% > set the page margins (2.5cm minimum in every side, as per IST rules)
%
\usepackage{geometry}	
\geometry{verbose,tmargin=2.5cm,bmargin=2.5cm,lmargin=2.5cm,rmargin=2.5cm}
%
% > allow setting line spacing (line spacing of 1.5, as per IST rules)
%
\usepackage{setspace}
\renewcommand{\baselinestretch}{1.5}
% ----------------------------------------------------------------------
% Include external packages.
\usepackage{graphicx}
\usepackage{amsmath}  % AMS mathematical facilities for LaTeX.
\usepackage{amsthm}   % Typesetting theorems (AMS style).
\usepackage{amsfonts} % 
\usepackage{subfigure}
\usepackage{subfigmat}
\usepackage{dcolumn}
\newcolumntype{d}{D{.}{.}{-1}} % column aligned by the point separator '.'
\newcolumntype{e}{D{E}{E}{-1}} % column aligned by the exponent 'E'
\usepackage[pdftex]{hyperref} % enhance documents that are to be
                              % output as HTML and PDF
\hypersetup{colorlinks,       % color text of links and anchors,
                              % eliminates borders around links
            linkcolor=blue,  % color for normal internal links
            anchorcolor=black,% color for anchor text
            citecolor=cyan,  % color for bibliographical citations
            filecolor=black,  % color for URLs which open local files
            menucolor=black,  % color for Acrobat menu items
            urlcolor=teal,   % color for linked URLs
	        bookmarksopen=true,    % don't expand bookmarks
	        bookmarksnumbered=true, % number bookmarks
            }
\usepackage[figure,table]{hypcap}
\usepackage[format=hang,labelfont=bf,font=small]{caption} 
\usepackage{cite}
\usepackage[printonlyused]{acronym}
\usepackage{lipsum}

% -----------------------------------------------------------------------------
% PACKAGE Cleveref:
% -----------------------------------------------------------------------------
% Clever Referencing of document parts
% Note: portuguese is supported through "brazilian" option
\usepackage[\IfLanguageName{english}{english}{brazilian}]{cleveref}
% -----------------------------------------------------------------------------
% PACKAGES xcolor, color
% -----------------------------------------------------------------------------
% These packages are required for list code snippets.
\usepackage{xcolor}
\usepackage{color}
% The following special color definitions are used in the IST Thesis
\definecolor{forestgreen}{RGB}{34,139,34}
\definecolor{orangered}{RGB}{239,134,64}
\definecolor{lightred}{rgb}{1,0.4,0.5}
\definecolor{orange}{rgb}{1,0.45,0.13}	
\definecolor{darkblue}{rgb}{0.0,0.0,0.6}
\definecolor{lightblue}{rgb}{0.1,0.57,0.7}
\definecolor{gray}{rgb}{0.4,0.4,0.4}
\definecolor{lightgray}{rgb}{0.95, 0.95, 0.95}
\definecolor{darkgray}{rgb}{0.4, 0.4, 0.4}
\definecolor{editorGray}{rgb}{0.95, 0.95, 0.95}
\definecolor{editorOcher}{rgb}{1, 0.5, 0} % #FF7F00 -> rgb(239, 169, 0)
\definecolor{chaptergrey}{rgb}{0.6,0.6,0.6}
\definecolor{editorGreen}{rgb}{0, 0.5, 0} % #007C00 -> rgb(0, 124, 0)
\definecolor{olive}{rgb}{0.17,0.59,0.20}
\definecolor{brown}{rgb}{0.69,0.31,0.31}
\definecolor{purple}{rgb}{0.38,0.18,0.81}




\usepackage{listings}
\lstset{escapeinside={<@}{@>}}
\usepackage{minted}

\lstdefinestyle{commandline} {%
language={[WinXP]command.com},
breaklines=true,
%aboveskip=\baselineskip,
belowskip=\baselineskip,
showstringspaces=false,
backgroundcolor=\color{black},
basicstyle=\small\color{white}\ttfamily
showstringspaces=false,
keywordstyle=\color{cyan}\bfseries,
stringstyle=\color{gray}\ttfamily,
commentstyle=\color{green}\itshape,
moredelim=[s][\color{yellow}\bfseries]{C:}{\>}
}

\lstdefinestyle{Bash} {%
language=bash,
breaklines=true,
belowskip=\baselineskip,
backgroundcolor=\color{lightgray},
showstringspaces=false,
keywordstyle=\color{black}\bfseries,
basicstyle=\small\color{black}\ttfamily,
stringstyle=\color{editorOcher}\ttfamily,
commentstyle=\color{brown}\itshape,
otherkeywords={xcode-select, mkdir,rm},
moredelim=[s][\color{darkblue}]{~$},
literate={~} {$\sim$}{1}
}

\lstdefinestyle{Rubytext} {%
language=Ruby,
breaklines=true,
belowskip=\baselineskip,
basicstyle=\small\ttfamily\color{black},
backgroundcolor=\color{lightgray},
showstringspaces=false,
commentstyle = \ttfamily\color{red},
keywordstyle=\ttfamily\color{blue},
stringstyle=\color{orange}
}
\lstset{escapeinside={<@}{@>}}

\lstdefinestyle{py} {%
	language=python,
	literate=%
	*{0}{{{\color{red}0}}}1
	{1}{{{\color{red}1}}}1
	{2}{{{\color{red}2}}}1
	{3}{{{\color{red}3}}}1
	{4}{{{\color{red}4}}}1
	{5}{{{\color{red}5}}}1
	{6}{{{\color{red}6}}}1
	{7}{{{\color{red}7}}}1
	{8}{{{\color{red}8}}}1
	{9}{{{\color{red}9}}}1,
	basicstyle=\small\ttfamily,
	numbers=left,
	% numberstyle=\tiny,
	% stepnumber=2,
	numbersep=5pt,
	tabsize=4,
	extendedchars=true,
	breaklines=true,
	keywordstyle=\color{blue}\bfseries,
	frame=b,
	commentstyle=\color{brown}\itshape,
	stringstyle=\color{editorOcher}\ttfamily,
	showspaces=false,
	showtabs=false,
	xleftmargin=17pt,
	framexleftmargin=17pt,
	framexrightmargin=5pt,
	framexbottommargin=4pt,
	backgroundcolor=\color{lightgray},
	showstringspaces=false,
}










% DEFINE COMMAND FOR: Degree Title depending on language
\newcommand{\tlangDegree}{\IfLanguageName{english}{Computer Science and Engineering}{Engenharia Informática e de Computadores}}
% DEFINE COMMAND FOR: Course Title depending on language
\newcommand{\tlangCourse}{\IfLanguageName{english}{Databases}{Bases de Dados}}
%%%%%%%%%%%%%%%%%%%%%%%%%%%%%%%%%%%%%%%%%%%%%%%%%%%%%%%%%%%%%%%%%%%%%%%%
