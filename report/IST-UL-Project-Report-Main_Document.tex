% #############################################################################    % This is the MAIN DOCUMENT of the IST-UL-Project-Report TEMPLATE. 
% !TEX root = ./main.tex
% #############################################################################
% The document is automatically set for english or portuguese by just selecting
% the MAIN LANGUAGE in file 'IST-UL-Project-Report-Preamble.tex' 
% #############################################################################
% Version 1.0, October 2018
% BY: Prof. Rui Santos Cruz, rui.s.cruz@tecnico.ulisboa.pt
% #############################################################################
% Set the document class
% ----------------------------------------------------------------------
\documentclass[12pt,a4paper]{report}
\usepackage{latexsym}
\usepackage{dirtree}
\usepackage{float}



% -----------------------------------------------------------------------------
% The Preamble document contains all the necessary Packages for typesetting
% Modify it to suit your needs
% -----------------------------------------------------------------------------
\input{./IST-UL-Project-Report-Preamble}

\begin{document}

% Set plain page style (no headers, footer with centered page number)
\pagestyle{plain}
% Set roman numbering (i,ii,...) before the start of chapters
\pagenumbering{roman}
% ----------------------------------------------------------------------------
% Cover page
% #############################################################################     % This is the FRONT COVER of the IST-UL-Project-Report TEMPLATE. 
% !TEX root = ./main.tex
% #############################################################################
% Version 1.0, October 2018
% BY: Prof. Rui Santos Cruz, rui.s.cruz@tecnico.ulisboa.pt
% #############################################################################
% #############################################################################
% DO NOT CHANGE THE FOLLOWING 4 LINES
\thispagestyle {empty}
\includegraphics[width=5cm]{./pictures/IST_A_RGB_POS.png}
\begin{center}
\vspace{5.0cm}
% #############################################################################
% #############################################################################
%
% INSERT THE TITLE OF THE PROJECT HERE
{\FontLb DB Project} \\
\vspace{0.2cm}
%
% INSERT THE SUBTITLE OF THE REPORT HERE
{\FontMn Part 2} \\
\vspace{1.0cm}
{\FontLn \tlangCourse} \\
\vspace{0.3cm}
{\FontMn Prof. Ana Cláudia Madeira David} \\
\vspace{0.5cm}
\textbf{Group nr.:}: 177 \\
\begin{center}
\begin{table}[h!]
\centering
\textbf{Total Effort}: 21 hours \\
\vspace{0.6cm}
\begin{tabular}{||c c c||} 
 \hline
 Student's Number & Full Name & Relative effort  \\ [0.5ex] 
 \hline\hline
       96098: & Tomás Gonçalves Lopes Costa Carvalho & 33\% \\
       99078: & Guilherme Henrique Corrêa Carabalone & 33\% \\
       99095: & João Paulo Melo Furtado & 33\% \\

 \hline
\end{tabular}
\caption{Students from the 'BDL05' shift.}
\end{table}
\end{center}
% \begin{center}
% \begin{tabular}{r@{~}l l}
%     \multicolumn{3}{c}{\bfseries\textbf{ }} \\
%     % INSERT YOUR TEAM NUMBER HERE
%     & \textbf{Grupo nr.}: & 177 \\
%     & \textbf{Turno nr.}: & 5 \\
%     % INSERT IDs and NAMES of STUDENTS
    
%     & Número de Aluno & Nome & Esforço Relativo \\
%     & 96098: & Tomás Gonçalves Lopes Costa Carvalho \\
%     & 99078: & Guilherme Henrique Corrêa Carabalone \\
%     & 99095: & João Paulo Melo Furtado \\
% \end{tabular}
% \end{center}
\vspace{2.0cm}
{\FontMb \tlangDegree} \\
{\FontMb IST-TAGUSPARK} \\
\vspace{1.5cm}
{\FontMb 2021/2022} \\
\end{center}
\cleardoublepage
% ----------------------------------------------------------------------------
% Table of contents,nn list of tables, list of figures and nomenclature
% ----------------------------------------------------------------------------
% Set arabic numbering (1,2,...) after preface
\setcounter{page}{1}
\pagenumbering{arabic}
% #############################################################################
% 
% BEGIN MAIN DOCUMENT BODY
% 
% #############################################################################

\section*{Application}

The app can be found over this link: \url{https://web.ist.utl.pt/ist199078/app.cgi/}.
The file structure:
\dirtree{%
.1 web/.
.2 app.cgi.
.2 templates/.
.3 add\_category.html.
.3 add\_retailer.html.
.3 add\_subcat.html.
.3 category.html.
.3 error.html.
.3 ivm.html.
.3 replenishments.html.
.3 retailer.html.
.3 subcat.html.
}

\section*{Archive's relations}
The purpose of each file:

\renewcommand{\arraystretch}{1.6}

\begin{table}[H]
\centering
\begin{tabular}{{p{0.2\linewidth} | p{0.8\linewidth}}}
\multicolumn{2}{c}{CGI}                                                                                                                   \\
app.cgi             & The script executed by the web client, containing render instructions, SQL instructions and the routing of the app. \\
\multicolumn{2}{c}{Templates}                                                                                                             \\
add\_category.html  & Template to add a new super category.                                                                               \\
add\_retailer.html  & Template to add a new retailer.                                                                                     \\
category.html       & Template to list all the categories, both simple and super.                                                         \\
error.html          & Template to generate an error.                                                                                      \\
ivm.html            & Template to list all the IVMs.                                                                                      \\
replenishments.html & Template to list all the replenishments of an IVM.                                                                  \\
retailer.html       & Template to list all retailers.                                                                                     \\
subcat.html         & Template to list all the categories that are categories of the argument(the category chosen).                      
\end{tabular}
\end{table}


\newpage


\section*{Indexes}
Point 7 from the Databases 3rd delivery draft:

\begin{verbatim}


7.1)

  CREATE INDEX index\_cat\_name ON responsable\_for(category)


\end{verbatim}

It is not necessary to generate an index for TIN, since it is a primary key of retailer and foreignkey of responsible\_for.
`P.nome\_cat = 'Frutos'` has a high selectivity pick.


\begin{verbatim}


7.2)

  CREATE INDEX index\_cat\_name ON product(product\_category)
  CREATE INDEX index\_product\_descr ON product(descr)
  CREATE INDEX index\_cat\_name ON has\_category(category\_name)


\end{verbatim}

It is necessary to generate an index for both product\_category and category\_name because they are foreign keys on other tables that are not product nor has\_catagory.
We know that the description of the product starts with 'A', so makes sense to use the index.
\end{document}
